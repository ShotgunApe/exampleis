%!TEX root = ../username.tex
\chapter{Introduction}\label{intro}
So why would you want to use \lt instead of Microsoft Word\texttrademark? I can think of several reasons. The main one for this author is that \lt takes care of all of the numbering automatically. This means that if you decide to rearrange material in your IS, you do not have to worry about renumbering or references. This makes it very easy to play around with the structure of your thesis. The second reason is that it is ultimately faster than Word\texttrademark. How? Well, after a week or so of using \lt you will begin to remember the commands that you use frequently and won't have to use the \lt pallet in TeXShop or TeXnicCenter. So you can just type everything including the mathematics, where with \msw you would have to use the Equation Editor.

I have also tried to make things more efficient by organizing the example folder as follows. There is a \texttt{username.tex} file which you will want to rename using your username and which is what you will enter all of the information about your IS into. \texttt{username.tex} also has explanations about other files that you might need to edit. In addition there are folders for chapters, appendices, styles, and figures. This structure is there to try and reduce file clutter and to help you stay organized. There should also be a .bib file which you can use as a model for your own .bib file. The .bib file has your bibliographic information.

\lt is really easy to learn. For an average IS, the author will only need to learn a handful of commands. For this small bit of effort, you get a tremendous amount of flexibility and a very beautiful document. The following chapters will introduce some of the common things a student might need to do in a thesis.