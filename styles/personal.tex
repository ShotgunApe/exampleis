%%%%%%%%%%%%%%%%%%%%%%%%%%%%%%%%%%%%%%%%%%%%%%%%%%%%%%%%%%%%%%%%%%%%%%%%%%%%%%%%%%%%%%%%%%%%%%
%
%                                                       Personal Commands
%                                                                    
% There will be certain commands that you use frequently in the thesis. You can give these
% commands new names which are easier for you to remember. You can also combine several
% commands into a new command of your own. See The LaTeX Companion or Guide to LaTeX for
% examples on defining your own commands. These are commands that I defined to cut down on typing.
%
%%%%%%%%%%%%%%%%%%%%%%%%%%%%%%%%%%%%%%%%%%%%%%%%%%%%%%%%%%%%%%%%%%%%%%%%%%%%%%%%%%%%%%%%%%%%%%

\newcommand{\fl}{\ell}
\newcommand{\lt}{\LaTeX\ }
\newcommand{\msw}{Word\texttrademark\ }
\newcommand{\xt}{\ifthenelse{\boolean{xetex}}{\XeTeX\ }{XeTeX} }
%\newcommand{\Cl}{\ensuremath{\textup{C}_\fl}}
%\newcommand{\bCl}{C$_{\ell}$}
%\newcommand{\Al}{\ensuremath{\textup{A}_\fl}}
%\newcommand{\msum}{{(m_1+\cdots+m_\ell)}}
%\newcommand{\Nsum}{{(N_1+\cdots+N_\ell)}}
%\newcommand{\ysum}{{(y_1+\cdots+y_\ell)}}
%\newcommand{\Nsub}{{N_1+\cdots+N_\ell}}
%\newcommand{\ysub}{{y_1+\cdots+y_\ell}}
%\newcommand{\xsub}{{x_1+\cdots+x_\ell}}
%\newcommand{\ysqsum}{{y_1^2+\cdots +y_{\fl}^2}}
%\newcommand{\msqsum}{{m_1^2+\cdots +m_{\fl}^2}}
%\newcommand{\ratio}{\left(\frac{\beta}{\alpha}\right)}
%\newcommand{\LT}{\ensuremath{\LaTeX{}}}

%%%%%%%%%%%%%%%%%%%%%%%%%%%%%%%%%%%%%%%%%%%%%%%%%%%%%%%%%%%%%%%%%%%%%%%%%%%%%%%%%%%%%%%%%%%%%%
% These commands have one argument and are entered as \commandname{argument}.
%%%%%%%%%%%%%%%%%%%%%%%%%%%%%%%%%%%%%%%%%%%%%%%%%%%%%%%%%%%%%%%%%%%%%%%%%%%%%%%%%%%%%%%%%%%%%%

%\newcommand{\bd}[1]{\textbf{#1}}
\newcommand{\mbd}[1]{{\mathbf{#1}}}
%\newcommand{\abs}[1]{\vert{#1}\vert}
\newcommand{\bvec}[1]{{\mbd{#1}}}
%\newcommand{\lvec}[1]{\abs{\bvec{#1}}}
%\newcommand{\nesmallprod}[1]{\prod_{\substack{#1=1\\
%#1\neq p}}^{\fl}}
%\newcommand{\esec}[1]{e_{2}({#1}_1,\ldots ,{#1}_\fl)}
%\newcommand{\smallprod}[1]{\prod_{#1=1}^{\fl}}
%\newcommand{\incsum}[1]{{#1}_2+2{#1}_3+\cdots +(\fl -1){#1}_\fl}
%\newcommand{\binomsum}[1]{\binom{{#1}_1}{2}+\cdots +\binom{{#1}_\fl}{2}}
%\newcommand{\imultsum}[1]{\multsum{{#1}_k\ge 0}{k=1,\ldots ,\fl}}
%\newcommand{\diagsum}[1]{\sum _{\substack{{#1}_k\ge 0\\
%k=1, \ldots ,\fl\\
%\lvec{#1}=m}}}
%\newcommand{\Mb}[1][\fl]{\ensuremath{\textup{\bd{M}}_b^{(#1)}}}
%\newcommand{\HLV}[1]{\ensuremath{\textup{\bd{H}}_{#1}}}
%\newcommand{\Rq}[1][p]{\ensuremath{\textup{R}_q^{(#1)}}}
\newcommand{\degree}[1]{\ensuremath{#1^{\circ}}}
\newcommand{\ip}[1]{\texttt{#1}\index{packages!#1}}
\newcommand{\ic}[1]{\texttt{$\backslash$#1}\index{commands!#1}}
\newcommand{\ie}[1]{#1\index{#1}}

%%%%%%%%%%%%%%%%%%%%%%%%%%%%%%%%%%%%%%%%%%%%%%%%%%%%%%%%%%%%%%%%%%%%%%%%%%%%%%%%%%%%%%%%%%%%%%
% These commands have 2 or more arguments some with default values for the first argument. You
% can learn a lot about constructing complicated equations by studying the commands in this %section.
%%%%%%%%%%%%%%%%%%%%%%%%%%%%%%%%%%%%%%%%%%%%%%%%%%%%%%%%%%%%%%%%%%%%%%%%%%%%%%%%%%%%%%%%%%%%%%

%\newcommand{\qbinom}[2]{\ensuremath{\left[{#1}\atop{#2}\right]_q}}
%\newcommand{\sqprod}[2]{\prod_{#1,#2=1}^{\fl}}
%\newcommand{\triprod}[2]{\prod_{1\le #1<#2\le \fl}}
%\newcommand{\nesqprod}[2]{\prod_{\substack{#1,#2=1\\
%#1,#2\neq p}}^{\fl}}
%\newcommand{\netriprod}[2]{\prod_{\substack{1\le #1<#2\le \fl\\
%#1,#2\neq p}}}
\newcommand{\qrfac}[3][\ ]{\left({#2}\right)_{#3}^{#1}}
%\newcommand{\multsum}[2]{\sum_{\substack{{#1}\\
%\\
%{#2}}}}
%\newcommand{\fmultsum}[2][N]{\multsum{0\le {{#2}_k}\le {{#1}_k}}{k=1,\ldots ,\fl}}
%\newcommand{\pq}[2]{\ _{#1}\varphi_{#2}}
%\newcommand{\mess}[2][y_k]{\frac{\qrfac{\alpha x_k}{#2}\qrfac{qx_k\beta^{-1}}{#2}}{\qrfac{\beta x_k}{#1}
%\qrfac{qx_k\alpha^{-1}}{#1}}}
%\newcommand{\MG}[7][\fl]{\ensuremath{\left[\textup{MG}\right]_{#2}^{(#1)}{#3}q;{#4};{#5}^{#6}{#7}}}

%%%%%%%%%%%%%%%%%%%%%%%%%%%%%%%%%%%%%%%%%%%%%%%%%%%%%%%%%%%%%%%%%%%%%%%%%%%%%%%%%%%%%%%%%%%%%%
% These commands define new environments
%%%%%%%%%%%%%%%%%%%%%%%%%%%%%%%%%%%%%%%%%%%%%%%%%%%%%%%%%%%%%%%%%%%%%%%%%%%%%%%%%%%%%%%%%%%%%%

\newcounter{unnumft}
\setcounter{unnumft}{0}
\newenvironment{unnumft}[2]{\renewcommand{\thefootnote}{}\footnote{#1}\footnote{#2}} {\addtocounter{footnote}{-2}}
\newenvironment{wooexample}{\small
\begin{singlespace}
\begin{example}}{\end{example}
\end{singlespace}}

\graphicspath{{./figures/}}% for setting where to look for figures
%\citestyle{wooster}% change the style of citations. Math and CS people should leave this alone.

%%%%%%%%%%%%%%%%%%%%%%%%%%%%%%%%%%%%%%%%%%%%%%%%%%%%%%%%%%%%%%%%%%%%%%%%%%%%%%%%%%%%%%
% Modify the formatting of the back references
%%%%%%%%%%%%%%%%%%%%%%%%%%%%%%%%%%%%%%%%%%%%%%%%%%%%%%%%%%%%%%%%%%%%%%%%%%%%%%%%%%%%%%
\DefineBibliographyStrings{english}{%
	backrefpage  = {page }, % for single page number
	backrefpages = {pages } % for multiple page numbers
}




